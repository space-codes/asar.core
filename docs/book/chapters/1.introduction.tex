\chapter{Introduction}
\pagenumbering{arabic}
% This chapter is to introduce your project, justify the need for it and explain the steps you follow to achieve it. The main outcomes from the project should be clearly stated. The organization of the document should finally be given

% In this space, before the first section, write a general introduction to the project

Historical Arabic manuscript documents are very important in particular for historians, sociologists, researchers, and students. These documents contain very precious knowledge, which is available for consultation through digital libraries, especially the Islamic manuscripts that are untapped sources of rich Islamic heritage. The information retrieval in these libraries is very difficult because of the difficulty of the manual search in an enormous set of digital pages (images). Hence the need to make use of \acrfull{ocr} systems or any other methods in historical documents to facilitate this task. \\

Since the automatic recognition of Arabic handwritten in historical documents is a challenging problem and difficult task and historical handwritten document images are different from the modern document images by their loosely layout format in which
\begin{itemize}[itemsep=1pt, topsep=5pt]
    \item They contain overlapping components in a line.
    \item They include holes, spots, ornamentation or seals that degrade overall quality significantly.
    \item Also, the words have writer dependent varying shapes.
\end{itemize}

Due to all these problems, \acrfull{ocr} is inefficient and performed badly when applied to handwritten historical documents \cite{Kassis2016AutomaticSO}. Since pattern recognition field is being directed towards the digitization of handwritten and historical manuscripts with a view to preserving the heritage containing valuable information. We use word spotting which is a technique in patter recognition field that aims to locate in a target document, regions that are most similar to query word without recognizing the characters. \\

Word spotting methods are classified in two main categories, learning based and template matching based methods. The first type is inspired by OCR; where models of keywords are trained using labeled data which are used to recognize queries in the target document \cite{GhilasKeyPoints}. These methods allow string querying but they suffer from the need for large labeled databases to train the system and the user is limited to choose the queries in a finite vocabulary. \\

\acrfull{asar} is an intelligent system that provides digitizing services for historical Arabic manuscripts images by using word spotting and recognition techniques. It helps scientists and Arabic researchers to verify Arabic manuscripts to retrieve valuable information.

\section{Motivation and Justification}
Many researchers have problems identifying words in these manuscripts, and these problems may be time, financial, and physical health costs. Therefore, when we looked at the problems faced by these researchers that make them spend all this effort and exhaustion in interpreting and understanding the manuscripts, we found that the manuscripts contain a number of problems that make the task difficult for these researchers, including: handwritten, there are marginal notes, the lines are not straight, font sizes are different, different types of fonts, faded ink, decorations, the presence of non-rectangular areas, and paper wear problems because of its age. \\

\noindent
These researchers have a very hard time dealing with the Arabic manuscripts for extracting the information from them manually. Hence we need to build an intelligent system that helps them in recognizing these images and to enrich our Arabic and Islamic heritage. \\

\noindent
In \acrshort{asar}, we found software solutions to address all these problems by using modern technology to produce an intelligent system that helps Arab people to understand and verify historical manuscripts. Which it can upload the manuscript image and get the content of the image in digital format.


\section{Project Objectives and Problem Definition}
We aim to build an intelligent system that provides accurate analysis for historical Arabic manuscripts to be extracted into digital format for later prepossessing from the users. \\

\noindent
Since there are no previous experiments on word spotting methods applied in the Arabic language, then our problem based on pattern recognition techniques for handwritten Arabic manuscripts to understand the Arabic language shapes. In addition, the project also depends on the page, line, and word segmentation for unseen images for extracting each word as a cropped image for applying the recognition process.

\section{Project Outcomes}
The outcome of the project is web and mobile applications that lets the user to upload a manuscript image having difficult language styling to be analyzed by the system, then the output is a text result shown having the content of the image in a digital text which can copy or download as PDF file. Also, the system allows the user to apply the recognition process as an anonymous person without creating an account but it will save any previous experiments if the user has an account for later exploring and downloading.

\clearpage

\section{Project Scope}
The scope of this project is to help Arab people to extract valuable information from historical Arabic and Islamic manuscripts. These people can be

\begin{itemize}[itemsep=1pt, topsep=5pt]
    \item The manuscript verification researcher that his job to read, understand and digitize these manuscripts.
    \item Universities, Master and PhD students that work and publish heritage books.
    \item Also, the new researchers to continue working on previous works for particular science.
\end{itemize}

The system can help them for digitizing these manuscript images into text format.

\section{Document Organization}
In this chapter, we provided an introduction to the problem and the objectives of our intelligent system showing us the motivation to helping the Arabic community, \acrshort{asar}. The rest of the document is organized as follows:
\begin{itemize}[itemsep=1pt, topsep=5pt]
    \item In chapter 2 we review different approaches and methods to our similar problem, most of them are academically published papers as the methods used by private companies are not disclosed out to the public, then goes through all the necessary background needed to understand the rest of the report.
    \item In chapter 3 we explain how our system works. We dive deep into the architecture of \acrshort{asar} and discover what every module contributes including the theoretical parts.
    \item In chapter 4 we explain the implementation details for our system including technical parts, data design, and requirements of the system.
    \item In chapter 5 we explain how we tested different aspects of our system, and how it compares to related work.
    \item In chapter 6 we conclude the document by describing the challenges we faced, the lessons that we learned, and what the future may look like for \acrshort{asar}.
\end{itemize}

Afterward, the appendices outline the development platforms and tools that we used, and the user guide.
